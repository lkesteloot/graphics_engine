\begin{table}[ht]\small
\begin{tabular}{llll}
Instruction & Operands & Format & Description\\\hline
\tt LD&\tt r1, r2(imm)&RA&Load Register\\
\tt ST&\tt r2(imm), r1&AR&Save Register\\
\tt LI&\tt r1, r2(imm)&RA&Load immediate\medskip\\
\tt ADD&\tt r1, r2, r3&RRR&Fixed-point Add\\
\tt SUB&\tt r1, r2, r3&RRR&Fixed-point Subtract\\
\tt MUL&\tt r1, r2, r3&RRR&Integer Signed Multiplication\\
\tt MULF&\tt r1, r2, r3&RRR&Fixed-point Signed Multiplication\\
\tt RECP&\tt r1, r2&RR&Fixed-point Signed Reciprocal\\
\tt AND&\tt r1, r2, r3&RRR&Bit-wise And\\
\tt OR&\tt r1, r2, r3&RRR&Bit-wise Or\\
\tt XOR&\tt r1, r2, r3&RRR&Bit-wise Xor\\
\tt NOT&\tt r1, r2&RR&Bit-wise Not\\
\tt LSL&\tt r1, r2, r3&RRR&Logical Shift Left\\
\tt LSR&\tt r1, r2, r3&RRR&Logical Shift Right\\
\tt ASR&\tt r1, r2, r3&RRR&Arithmetic Shift Right\\
\tt LSLI&\tt r1, r2, imm&RRI&Logical Shift Left Immediate\\
\tt LSRI&\tt r1, r2, imm&RRI&Logical Shift Right Immediate\\
\tt ASRI&\tt r1, r2, imm&RRI&Arithmetic Shift Right Immediate\medskip\\
\tt JMP&\tt r2(imm)&A&Unconditional Jump\\
\tt CALL&\tt r2(imm)&A&Unconditional Subroutine Call\\
\tt JEQ&\tt r2(imm)&A&Jump If Equal\\
\tt JEQL&\tt r2(imm)&A&Jump If Equal Likely\\
\tt JNE&\tt r2(imm)&A&Jump If Not Equal\\
\tt JNEL&\tt r2(imm)&A&Jump If Not Equal Likely\\
\tt JLT&\tt r2(imm)&A&Jump If Less Than\\
\tt JLTL&\tt r2(imm)&A&Jump If Less Than Likely\\
\tt JLE&\tt r2(imm)&A&Jump If Less Than or Equal To\\
\tt JLEL&\tt r2(imm)&A&Jump If Less Than or Equal To Likely\\
\tt JGT&\tt r2(imm)&A&Jump If Greater Than\\
\tt JGTL&\tt r2(imm)&A&Jump If Greater Than Likely\\
\tt JGE&\tt r2(imm)&A&Jump If Greater Than or Equal To\\
\tt JGEL&\tt r2(imm)&A&Jump If Greater Than or Equal To Likely\medskip\\
\tt DOT&\tt r1, r2, r3&RRR&Fixed-point Vector Dot-Product\medskip\\
\tt READ&\tt r1, r2&RR&DMA Read from External Memory\\
\tt WRITE&\tt r1, r2&RR&DMA Write to External Memory\medskip\\
\tt SCRSET&\tt r1, r2, imm&RRI&Set Screen Location\\
\tt CLS&\tt r1, r2&RR&Clear Screen\\
\tt TREESET&\tt r1, r2, r3&RRR&Setup Tree and Clear Enable Flags\\
\tt TREEIN&\tt r1, r2&RR&Edge Equation Test\\
\tt TREEVAL&\tt r1, r2, imm&RRI&Parameter Interpolation\\
\tt TREEPUT&\tt &OP&Tree Write to Frame Buffer\\
\end{tabular}
\caption{Instruction Set Summary}
\label{tab:instsummary}
\end{table}

\Section{Register Loading and Storing}
\noindent\textsf{\textbf{\Large LD}}\index{instruction set!LD@{\tt LD}}\par
\noindent {\sc Load Register}\par\begin{indented}{\bf Format:}
{\tt LD r1, r2(imm)}\par\vspace{3ex}
\formatRA{000000}{LD}
\end{indented}\vspace{4ex}
\begin{indented}{\bf Description:}
The word in data memory at location {\tt r2(imm)} is loaded into
register {\tt r1}.
\end{indented}
\begin{indented}{\bf Operation:}\vspace{.8ex}
\begin{verbatimtab}
reg[r1] = dMem[imm + reg[r2]];
SetStatusWord (reg[r1]);
\end{verbatimtab}
\end{indented}
\vspace{2em}

\noindent\textsf{\textbf{\Large ST}}\index{instruction set!ST@{\tt ST}}\par
\noindent {\sc Save Register}\par\begin{indented}{\bf Format:}
{\tt ST r2(imm), r1}\par\vspace{3ex}
\formatAR{000001}{ST}
\end{indented}\vspace{4ex}
\begin{indented}{\bf Description:}
The word in register {\tt r1} is stored in data memory at
location {\tt r2(imm)}.  Note that although the assembly operands are
the reverse of the {\tt LD} instruction, the instruction encoding does
not change.
\end{indented}
\begin{indented}{\bf Operation:}\vspace{.8ex}
\begin{verbatimtab}
dMem[imm + reg[r2]] = reg[r1];
\end{verbatimtab}
\end{indented}
\vspace{2em}

\newpage
\noindent\textsf{\textbf{\Large LI}}\index{instruction set!LI@{\tt LI}}\par
\noindent {\sc Load immediate}\par\begin{indented}{\bf Format:}
{\tt LI r1, r2(imm)}\par\vspace{3ex}
\formatRA{000010}{LI}
\end{indented}\vspace{4ex}
\begin{indented}{\bf Description:}
The 16-bit value {\tt r2(imm)} is loaded into {\tt r1}.  The upper 16
bits of {\tt imm} are sign-extended.  (I.e., all upper 16 bits are set
to bit 15.)
\end{indented}
\begin{indented}{\bf Operation:}\vspace{.8ex}
\begin{verbatimtab}
reg[r1] = (signed long)(signed short)(imm + reg[r2]);
SetStatusWord (reg[r1]);
\end{verbatimtab}
\end{indented}
\vspace{2em}

\Section{General Purpose Arithmetic and Logic}
\subsection{Arithmetic}
\noindent\textsf{\textbf{\Large ADD}}\index{instruction set!ADD@{\tt ADD}}\par
\noindent {\sc Fixed-point Add}\par\begin{indented}{\bf Format:}
{\tt ADD r1, r2, r3}\par\vspace{3ex}
\formatRRR{000011}{ADD}
\end{indented}\vspace{4ex}
\begin{indented}{\bf Description:}
Registers {\tt r2} and {\tt r3} are added together and the sum is placed
in {\tt r1}.
\end{indented}
\begin{indented}{\bf Operation:}\vspace{.8ex}
\begin{verbatimtab}
reg[r1] = reg[r2] + reg[r3];
SetStatusWord (reg[r1]);
\end{verbatimtab}
\end{indented}
\vspace{2em}

\newpage
\noindent\textsf{\textbf{\Large SUB}}\index{instruction set!SUB@{\tt SUB}}\par
\noindent {\sc Fixed-point Subtract}\par\begin{indented}{\bf Format:}
{\tt SUB r1, r2, r3}\par\vspace{3ex}
\formatRRR{000100}{SUB}
\end{indented}\vspace{4ex}
\begin{indented}{\bf Description:}
Register {\tt r3} is subtracted from register {\tt r3} and the difference is
placed in {\tt r1}.  If {\tt r0} is used as the destination, then this
instruction is effectively a {\tt CMP} (compare) instruction.  If {\tt r0}
is used for {\tt r2}, then this instruction is effectively a {\tt NEG}
(negate) instruction.
\end{indented}
\begin{indented}{\bf Operation:}\vspace{.8ex}
\begin{verbatimtab}
reg[r1] = reg[r2] - reg[r3];
SetStatusWord (reg[r1]);
\end{verbatimtab}
\end{indented}
\vspace{2em}

\noindent\textsf{\textbf{\Large MUL}}\index{instruction set!MUL@{\tt MUL}}\par
\noindent {\sc Integer Signed Multiplication}\par\begin{indented}{\bf Format:}
{\tt MUL r1, r2, r3}\par\vspace{3ex}
\formatRRR{000101}{MUL}
\end{indented}\vspace{4ex}
\begin{indented}{\bf Description:}
Registers {\tt r2} and {\tt r3} are multiplied together and the product is
placed in {\tt r1}.  The multiplication assumes that both operands are
signed in two's-complement notation, and the result is also signed in
two's-complement notation.  Overflow is ignored.
\end{indented}
\begin{indented}{\bf Operation:}\vspace{.8ex}
\begin{verbatimtab}
reg[r1] = (signed long)reg[r2] * (signed long)reg[r3];
SetStatusWord (reg[r1]);
\end{verbatimtab}
\end{indented}
\vspace{2em}

\newpage
\noindent\textsf{\textbf{\Large MULF}}\index{instruction set!MULF@{\tt MULF}}\par
\noindent {\sc Fixed-point Signed Multiplication}\par\begin{indented}{\bf Format:}
{\tt MULF r1, r2, r3}\par\vspace{3ex}
\formatRRR{000110}{MULF}
\end{indented}\vspace{4ex}
\begin{indented}{\bf Description:}
Registers {\tt r2} and {\tt r3} are multiplied together and the product is
placed in {\tt r1}.  The multiplication assumes that both operands are
signed in two's complement notation, and the result is also signed in
two's complement notation.  Operands and product are in {\tt s15.16}
fixed-point format.  (Actually, it really means that one operand is in
{\tt s15.16} and the other operand is anything.  The result is in the
same format as the other operand.)  Overflow is ignored.
\end{indented}
\begin{indented}{\bf Operation:}\vspace{.8ex}
\begin{verbatimtab}
reg[r1] = ((signed long)reg[r2] * (signed long)reg[r3]) >> 16;
SetStatusWord (reg[r1]);
\end{verbatimtab}
\end{indented}
\vspace{2em}

\noindent\textsf{\textbf{\Large RECP}}\index{instruction set!RECP@{\tt RECP}}\par
\noindent {\sc Fixed-point Signed Reciprocal}\par\begin{indented}{\bf Format:}
{\tt RECP r1, r2}\par\vspace{3ex}
\formatRR{000111}{RECP}
\end{indented}\vspace{4ex}
\begin{indented}{\bf Description:}
The reciprocal of register {\tt r2} is placed in {\tt r1}.  Both operands
and result are in two's complement notation.  The input is assumed to
be an integer (in {\tt s31.0} format) and the result is in {\tt s.31}
fixed-point format.
\end{indented}
\begin{indented}{\bf Operation:}\vspace{.8ex}
\begin{verbatimtab}
reg[r1] = 0x7FFFFFFF / (signed long)reg[r2];
SetStatusWord (reg[r1]);
\end{verbatimtab}
\end{indented}
\vspace{2em}

\newpage
\subsection{Logic and Bitwise}
\noindent\textsf{\textbf{\Large AND}}\index{instruction set!AND@{\tt AND}}\par
\noindent {\sc Bit-wise And}\par\begin{indented}{\bf Format:}
{\tt AND r1, r2, r3}\par\vspace{3ex}
\formatRRR{001000}{AND}
\end{indented}\vspace{4ex}
\begin{indented}{\bf Description:}
Registers {\tt r2} and {\tt r3} and bit-wise ANDed and the result is
placed in {\tt r1}.
\end{indented}
\begin{indented}{\bf Operation:}\vspace{.8ex}
\begin{verbatimtab}
reg[r1] = reg[r2] & reg[r3];
SetStatusWord (reg[r1]);
\end{verbatimtab}
\end{indented}
\vspace{2em}

\noindent\textsf{\textbf{\Large OR}}\index{instruction set!OR@{\tt OR}}\par
\noindent {\sc Bit-wise Or}\par\begin{indented}{\bf Format:}
{\tt OR r1, r2, r3}\par\vspace{3ex}
\formatRRR{001001}{OR}
\end{indented}\vspace{4ex}
\begin{indented}{\bf Description:}
Registers {\tt r2} and {\tt r3} and bit-wise ORed and the result is
placed in {\tt r1}.
\end{indented}
\begin{indented}{\bf Operation:}\vspace{.8ex}
\begin{verbatimtab}
reg[r1] = reg[r2] | reg[r3];
SetStatusWord (reg[r1]);
\end{verbatimtab}
\end{indented}
\vspace{2em}

\newpage
\noindent\textsf{\textbf{\Large XOR}}\index{instruction set!XOR@{\tt XOR}}\par
\noindent {\sc Bit-wise Xor}\par\begin{indented}{\bf Format:}
{\tt XOR r1, r2, r3}\par\vspace{3ex}
\formatRRR{001010}{XOR}
\end{indented}\vspace{4ex}
\begin{indented}{\bf Description:}
Registers {\tt r2} and {\tt r3} and bit-wise XORed and the result is
placed in {\tt r1}.
\end{indented}
\begin{indented}{\bf Operation:}\vspace{.8ex}
\begin{verbatimtab}
reg[r1] = reg[r2] ^ reg[r3];
SetStatusWord (reg[r1]);
\end{verbatimtab}
\end{indented}
\vspace{2em}

\noindent\textsf{\textbf{\Large NOT}}\index{instruction set!NOT@{\tt NOT}}\par
\noindent {\sc Bit-wise Not}\par\begin{indented}{\bf Format:}
{\tt NOT r1, r2}\par\vspace{3ex}
\formatRR{001011}{NOT}
\end{indented}\vspace{4ex}
\begin{indented}{\bf Description:}
The bit-wise inversion of register {\tt r2} is placed in {\tt r1}.
\end{indented}
\begin{indented}{\bf Operation:}\vspace{.8ex}
\begin{verbatimtab}
reg[r1] = ~reg[r2];
SetStatusWord (reg[r1]);
\end{verbatimtab}
\end{indented}
\vspace{2em}

\newpage
\noindent\textsf{\textbf{\Large LSL}}\index{instruction set!LSL@{\tt LSL}}\par
\noindent {\sc Logical Shift Left}\par\begin{indented}{\bf Format:}
{\tt LSL r1, r2, r3}\par\vspace{3ex}
\formatRRR{001100}{LSL}
\end{indented}\vspace{4ex}
\begin{indented}{\bf Description:}
Register {\tt r2} is shifted left {\tt r3} bits and the result is placed
in {\tt r1}.
\end{indented}
\begin{indented}{\bf Operation:}\vspace{.8ex}
\begin{verbatimtab}
reg[r1] = reg[r2] << reg[r3];
SetStatusWord (reg[r1]);
\end{verbatimtab}
\end{indented}
\vspace{2em}

\noindent\textsf{\textbf{\Large LSR}}\index{instruction set!LSR@{\tt LSR}}\par
\noindent {\sc Logical Shift Right}\par\begin{indented}{\bf Format:}
{\tt LSR r1, r2, r3}\par\vspace{3ex}
\formatRRR{001101}{LSR}
\end{indented}\vspace{4ex}
\begin{indented}{\bf Description:}
Register {\tt r2} is shifted right {\tt r3} bits and the result is placed
in {\tt r1}.  The bits shifted in from the left are zeros.
\end{indented}
\begin{indented}{\bf Operation:}\vspace{.8ex}
\begin{verbatimtab}
reg[r1] = reg[r2] >> reg[r3];
SetStatusWord (reg[r1]);
\end{verbatimtab}
\end{indented}
\vspace{2em}

\newpage
\noindent\textsf{\textbf{\Large ASR}}\index{instruction set!ASR@{\tt ASR}}\par
\noindent {\sc Arithmetic Shift Right}\par\begin{indented}{\bf Format:}
{\tt ASR r1, r2, r3}\par\vspace{3ex}
\formatRRR{001110}{ASR}
\end{indented}\vspace{4ex}
\begin{indented}{\bf Description:}
Register {\tt r2} is shifted right {\tt r3} bits and the result is placed
in {\tt r1}.  The bits shifted in from the left are the same as the sign
bit.
\end{indented}
\begin{indented}{\bf Operation:}\vspace{.8ex}
\begin{verbatimtab}
reg[r1] = (signed)reg[r2] >> reg[r3];
SetStatusWord (reg[r1]);
\end{verbatimtab}
\end{indented}
\vspace{2em}

\noindent\textsf{\textbf{\Large LSLI}}\index{instruction set!LSLI@{\tt LSLI}}\par
\noindent {\sc Logical Shift Left Immediate}\par\begin{indented}{\bf Format:}
{\tt LSLI r1, r2, imm}\par\vspace{3ex}
\formatRRI{001111}{LSLI}
\end{indented}\vspace{4ex}
\begin{indented}{\bf Description:}
Register {\tt r2} is shifted left {\tt imm} bits and the result is placed
in {\tt r1}.
\end{indented}
\begin{indented}{\bf Operation:}\vspace{.8ex}
\begin{verbatimtab}
reg[r1] = reg[r2] << imm;
SetStatusWord (reg[r1]);
\end{verbatimtab}
\end{indented}
\vspace{2em}

\newpage
\noindent\textsf{\textbf{\Large LSRI}}\index{instruction set!LSRI@{\tt LSRI}}\par
\noindent {\sc Logical Shift Right Immediate}\par\begin{indented}{\bf Format:}
{\tt LSRI r1, r2, imm}\par\vspace{3ex}
\formatRRI{010000}{LSRI}
\end{indented}\vspace{4ex}
\begin{indented}{\bf Description:}
Register {\tt r2} is shifted right {\tt imm} bits and the result is placed
in {\tt r1}.  The bits shifted in from the left are zeros.
\end{indented}
\begin{indented}{\bf Operation:}\vspace{.8ex}
\begin{verbatimtab}
reg[r1] = reg[r2] >> imm;
SetStatusWord (reg[r1]);
\end{verbatimtab}
\end{indented}
\vspace{2em}

\noindent\textsf{\textbf{\Large ASRI}}\index{instruction set!ASRI@{\tt ASRI}}\par
\noindent {\sc Arithmetic Shift Right Immediate}\par\begin{indented}{\bf Format:}
{\tt ASRI r1, r2, imm}\par\vspace{3ex}
\formatRRI{010001}{ASRI}
\end{indented}\vspace{4ex}
\begin{indented}{\bf Description:}
Register {\tt r2} is shifted right {\tt r3} bits and the result is placed
in {\tt r1}.  The bits shifted in from the left are the same as the sign
bit.
\end{indented}
\begin{indented}{\bf Operation:}\vspace{.8ex}
\begin{verbatimtab}
reg[r1] = (signed)reg[r2] >> imm;
SetStatusWord (reg[r1]);
\end{verbatimtab}
\end{indented}
\vspace{2em}

\newpage
\Section{Control Flow}
\noindent\textsf{\textbf{\Large JMP}}\index{instruction set!JMP@{\tt JMP}}\par
\noindent {\sc Unconditional Jump}\par\begin{indented}{\bf Format:}
{\tt JMP r2(imm)}\par\vspace{3ex}
\formatA{010010}{JMP}
\end{indented}\vspace{4ex}
\begin{indented}{\bf Description:}
Program execution will continue at address {\tt r2(imm)}.
\end{indented}
\begin{indented}{\bf Operation:}\vspace{.8ex}
\begin{verbatimtab}
pc = reg[r2] + imm;
\end{verbatimtab}
\end{indented}
\vspace{2em}

\noindent\textsf{\textbf{\Large CALL}}\index{instruction set!CALL@{\tt CALL}}\par
\noindent {\sc Unconditional Subroutine Call}\par\begin{indented}{\bf Format:}
{\tt CALL r2(imm)}\par\vspace{3ex}
\formatA{010011}{CALL}
\end{indented}\vspace{4ex}
\begin{indented}{\bf Description:}
The program counter is stored in register {\tt r31} and program execution
will continue at the address specified.
\end{indented}
\begin{indented}{\bf Operation:}\vspace{.8ex}
\begin{verbatimtab}
reg[31] = pc;
pc = reg[r2] + imm;
\end{verbatimtab}
\end{indented}
\vspace{2em}

\newpage
\noindent\textsf{\textbf{\Large JEQ}}\index{instruction set!JEQ@{\tt JEQ}}\par
\noindent {\sc Jump If Equal}\par\begin{indented}{\bf Format:}
{\tt JEQ r2(imm)}\par\vspace{3ex}
\formatA{010100}{JEQ}
\end{indented}\vspace{4ex}
\begin{indented}{\bf Description:}
Program execution will continue at the address specified if the {\tt Z} bit
of the status word is set.
\end{indented}
\begin{indented}{\bf Operation:}\vspace{.8ex}
\begin{verbatimtab}
if (sw & Z) {
    pc = reg[r2] + imm;
}
\end{verbatimtab}
\end{indented}
\vspace{2em}

\noindent\textsf{\textbf{\Large JEQL}}\index{instruction set!JEQL@{\tt JEQL}}\par
\noindent {\sc Jump If Equal Likely}\par\begin{indented}{\bf Format:}
{\tt JEQL r2(imm)}\par\vspace{3ex}
\formatA{010101}{JEQL}
\end{indented}\vspace{4ex}
\begin{indented}{\bf Description:}
Program execution will continue at the address specified if the {\tt Z} bit
of the status word is set.  The ``Likely'' is a hint to the implementation
that the jump is likely to take place and the instruction pipeline should
be fed from the destination address.  The semantics are the same as those
of {\tt JEQ}.
\end{indented}
\begin{indented}{\bf Operation:}\vspace{.8ex}
\begin{verbatimtab}
if (sw & Z) {
    pc = reg[r2] + imm;
}
\end{verbatimtab}
\end{indented}
\vspace{2em}

\newpage
\noindent\textsf{\textbf{\Large JNE}}\index{instruction set!JNE@{\tt JNE}}\par
\noindent {\sc Jump If Not Equal}\par\begin{indented}{\bf Format:}
{\tt JNE r2(imm)}\par\vspace{3ex}
\formatA{010110}{JNE}
\end{indented}\vspace{4ex}
\begin{indented}{\bf Description:}
Program execution will continue at the address specified if the {\tt Z} bit
of the status word is reset.
\end{indented}
\begin{indented}{\bf Operation:}\vspace{.8ex}
\begin{verbatimtab}
if (!(sw & Z)) {
    pc = reg[r2] + imm;
}
\end{verbatimtab}
\end{indented}
\vspace{2em}

\noindent\textsf{\textbf{\Large JNEL}}\index{instruction set!JNEL@{\tt JNEL}}\par
\noindent {\sc Jump If Not Equal Likely}\par\begin{indented}{\bf Format:}
{\tt JNEL r2(imm)}\par\vspace{3ex}
\formatA{010111}{JNEL}
\end{indented}\vspace{4ex}
\begin{indented}{\bf Description:}
Program execution will continue at the address specified if the {\tt Z} bit
of the status word is reset.  The ``Likely'' is a hint to the implementation
that the jump is likely to take place and the instruction pipeline should
be fed from the destination address.  The semantics are the same as those
of {\tt JNE}.
\end{indented}
\begin{indented}{\bf Operation:}\vspace{.8ex}
\begin{verbatimtab}
if (!(sw & Z)) {
    pc = reg[r2] + imm;
}
\end{verbatimtab}
\end{indented}
\vspace{2em}

\newpage
\noindent\textsf{\textbf{\Large JLT}}\index{instruction set!JLT@{\tt JLT}}\par
\noindent {\sc Jump If Less Than}\par\begin{indented}{\bf Format:}
{\tt JLT r2(imm)}\par\vspace{3ex}
\formatA{011000}{JLT}
\end{indented}\vspace{4ex}
\begin{indented}{\bf Description:}
Program execution will continue at the address specified if the {\tt N} bit
of the status word is set.
\end{indented}
\begin{indented}{\bf Operation:}\vspace{.8ex}
\begin{verbatimtab}
if (sw & N) {
    pc = reg[r2] + imm;
}
\end{verbatimtab}
\end{indented}
\vspace{2em}

\noindent\textsf{\textbf{\Large JLTL}}\index{instruction set!JLTL@{\tt JLTL}}\par
\noindent {\sc Jump If Less Than Likely}\par\begin{indented}{\bf Format:}
{\tt JLTL r2(imm)}\par\vspace{3ex}
\formatA{011001}{JLTL}
\end{indented}\vspace{4ex}
\begin{indented}{\bf Description:}
Program execution will continue at the address specified if the {\tt N} bit
of the status word is set.  The ``Likely'' is a hint to the implementation
that the jump is likely to take place and the instruction pipeline should
be fed from the destination address.  The semantics are the same as those
of {\tt JLT}.
\end{indented}
\begin{indented}{\bf Operation:}\vspace{.8ex}
\begin{verbatimtab}
if (sw & N) {
    pc = reg[r2] + imm;
}
\end{verbatimtab}
\end{indented}
\vspace{2em}

\newpage
\noindent\textsf{\textbf{\Large JLE}}\index{instruction set!JLE@{\tt JLE}}\par
\noindent {\sc Jump If Less Than or Equal To}\par\begin{indented}{\bf Format:}
{\tt JLE r2(imm)}\par\vspace{3ex}
\formatA{011010}{JLE}
\end{indented}\vspace{4ex}
\begin{indented}{\bf Description:}
Program execution will continue at the address specified if the {\tt N} and
{\tt Z} bits of the status word are set.
\end{indented}
\begin{indented}{\bf Operation:}\vspace{.8ex}
\begin{verbatimtab}
if (sw & (N | Z)) {
    pc = reg[r2] + imm;
}
\end{verbatimtab}
\end{indented}
\vspace{2em}

\noindent\textsf{\textbf{\Large JLEL}}\index{instruction set!JLEL@{\tt JLEL}}\par
\noindent {\sc Jump If Less Than or Equal To Likely}\par\begin{indented}{\bf Format:}
{\tt JLEL r2(imm)}\par\vspace{3ex}
\formatA{011011}{JLEL}
\end{indented}\vspace{4ex}
\begin{indented}{\bf Description:}
Program execution will continue at the address specified if the {\tt
N} and {\tt Z} bits of the status word are set.  The ``Likely'' is a
hint to the implementation that the jump is likely to take place and
the instruction pipeline should be fed from the destination address. 
The semantics are the same as those of {\tt JLE}.
\end{indented}
\begin{indented}{\bf Operation:}\vspace{.8ex}
\begin{verbatimtab}
if (sw & (N | Z)) {
    pc = reg[r2] + imm;
}
\end{verbatimtab}
\end{indented}
\vspace{2em}

\newpage
\noindent\textsf{\textbf{\Large JGT}}\index{instruction set!JGT@{\tt JGT}}\par
\noindent {\sc Jump If Greater Than}\par\begin{indented}{\bf Format:}
{\tt JGT r2(imm)}\par\vspace{3ex}
\formatA{011100}{JGT}
\end{indented}\vspace{4ex}
\begin{indented}{\bf Description:}
Program execution will continue at the address specified if the {\tt N} and
{\tt Z} bits of the status word are reset.
\end{indented}
\begin{indented}{\bf Operation:}\vspace{.8ex}
\begin{verbatimtab}
if (!(sw & (N | Z))) {
    pc = reg[r2] + imm;
}
\end{verbatimtab}
\end{indented}
\vspace{2em}

\noindent\textsf{\textbf{\Large JGTL}}\index{instruction set!JGTL@{\tt JGTL}}\par
\noindent {\sc Jump If Greater Than Likely}\par\begin{indented}{\bf Format:}
{\tt JGTL r2(imm)}\par\vspace{3ex}
\formatA{011101}{JGTL}
\end{indented}\vspace{4ex}
\begin{indented}{\bf Description:}
Program execution will continue at the address specified if the {\tt
N} and {\tt Z} bits of the status word are reset.  The ``Likely'' is a
hint to the implementation that the jump is likely to take place and
the instruction pipeline should be fed from the destination address. 
The semantics are the same as those of {\tt JGT}.
\end{indented}
\begin{indented}{\bf Operation:}\vspace{.8ex}
\begin{verbatimtab}
if (!(sw & (N | Z))) {
    pc = reg[r2] + imm;
}
\end{verbatimtab}
\end{indented}
\vspace{2em}

\newpage
\noindent\textsf{\textbf{\Large JGE}}\index{instruction set!JGE@{\tt JGE}}\par
\noindent {\sc Jump If Greater Than or Equal To}\par\begin{indented}{\bf Format:}
{\tt JGE r2(imm)}\par\vspace{3ex}
\formatA{011110}{JGE}
\end{indented}\vspace{4ex}
\begin{indented}{\bf Description:}
Program execution will continue at the address specified if the {\tt N} bit
of the status word is reset.
\end{indented}
\begin{indented}{\bf Operation:}\vspace{.8ex}
\begin{verbatimtab}
if (!(sw & N)) {
    pc = reg[r2] + imm;
}
\end{verbatimtab}
\end{indented}
\vspace{2em}

\noindent\textsf{\textbf{\Large JGEL}}\index{instruction set!JGEL@{\tt JGEL}}\par
\noindent {\sc Jump If Greater Than or Equal To Likely}\par\begin{indented}{\bf Format:}
{\tt JGEL r2(imm)}\par\vspace{3ex}
\formatA{011111}{JGEL}
\end{indented}\vspace{4ex}
\begin{indented}{\bf Description:}
Program execution will continue at the address specified if the {\tt N} bit
of the status word is reset.  The ``Likely'' is a hint to the implementation
that the jump is likely to take place and the instruction pipeline should
be fed from the destination address.  The semantics are the same as those
of {\tt JGE}.
\end{indented}
\begin{indented}{\bf Operation:}\vspace{.8ex}
\begin{verbatimtab}
if (!(sw & N)) {
    pc = reg[r2] + imm;
}
\end{verbatimtab}
\end{indented}
\vspace{2em}

\newpage
\Section{Special Purpose Arithmetic}
\noindent\textsf{\textbf{\Large DOT}}\index{instruction set!DOT@{\tt DOT}}\par
\noindent {\sc Fixed-point Vector Dot-Product}\par\begin{indented}{\bf Format:}
{\tt DOT r1, r2, r3}\par\vspace{3ex}
\formatRRR{100000}{DOT}
\end{indented}\vspace{4ex}
\begin{indented}{\bf Description:}
Registers {\tt r2} and {\tt r3} point to two four-element vectors in data
memory.  Each element is in {\tt s15.16} format.  The vectors' dot product,
also in {\tt s15.16} format, is stored in register {\tt r1}.
\end{indented}
\begin{indented}{\bf Operation:}\vspace{.8ex}
\begin{verbatimtab}
reg[r1] = 0;
for (i = 0; i < 4; i++) {
    reg[r1] += (dMem[reg[r2] + i] * dMem[reg[r3] + i]) >> 16;
}
\end{verbatimtab}
\end{indented}
\vspace{2em}

\Section{Direct Memory Access to External RAM}
\noindent\textsf{\textbf{\Large READ}}\index{instruction set!READ@{\tt READ}}\par
\noindent {\sc DMA Read from External Memory}\par\begin{indented}{\bf Format:}
{\tt READ r1, r2}\par\vspace{3ex}
\formatRR{100001}{READ}
\end{indented}\vspace{4ex}
\begin{indented}{\bf Description:}
The 16 words starting at location in external (shared) memory pointed to
by {\tt r2} are copied to the 16 words starting at location in data
memory pointed to by {\tt r1}.
\end{indented}
\begin{indented}{\bf Operation:}\vspace{.8ex}
\begin{verbatimtab}
for (i = 0; i < 16; i++) {
    dMem[reg[r1] + i] = eMem[reg[r2] + i];
}
\end{verbatimtab}
\end{indented}
\vspace{2em}

\newpage
\noindent\textsf{\textbf{\Large WRITE}}\index{instruction set!WRITE@{\tt WRITE}}\par
\noindent {\sc DMA Write to External Memory}\par\begin{indented}{\bf Format:}
{\tt WRITE r1, r2}\par\vspace{3ex}
\formatRR{100010}{WRITE}
\end{indented}\vspace{4ex}
\begin{indented}{\bf Description:}
The 16 words starting at location in data memory pointed to by {\tt
r2} are copied to the 16 words starting at location in external
(shared) memory pointed to by {\tt r1}.
\end{indented}
\begin{indented}{\bf Operation:}\vspace{.8ex}
\begin{verbatimtab}
for (i = 0; i < 16; i++) {
    eMem[reg[r1] + i] = dMem[reg[r2] + i];
}
\end{verbatimtab}
\end{indented}
\vspace{2em}

\Section{Rasterization}
\noindent\textsf{\textbf{\Large SCRSET}}\index{instruction set!SCRSET@{\tt SCRSET}}\par
\noindent {\sc Set Screen Location}\par\begin{indented}{\bf Format:}
{\tt SCRSET r1, r2, imm}\par\vspace{3ex}
\formatRRI{100011}{SCRSET}
\end{indented}\vspace{4ex}
\begin{indented}{\bf Description:}
Registers {\tt r1} and {\tt r2} contain the number of horizontal and
vertical pixels on the screen, respectively.  The word in data memory
pointed to by the immediate operand is used to determine the location
of the frame buffer in external memory.  The high half-word specifies
in kilobytes the starting address of the color frame buffer, and the
low half-word specifies in kilobytes the starting address of the Z
buffer.
\end{indented}
\begin{indented}{\bf Operation:}\vspace{.8ex}
\begin{verbatimtab}
width = reg[r1];
height = reg[r2];
fbAddr = (dMem[imm] >> 16) << 10;
zAddr = (dMem[imm] & 0xFFFF) << 10;
\end{verbatimtab}
\end{indented}
\vspace{2em}

\newpage
\noindent\textsf{\textbf{\Large CLS}}\index{instruction set!CLS@{\tt CLS}}\par
\noindent {\sc Clear Screen}\par\begin{indented}{\bf Format:}
{\tt CLS r1, r2}\par\vspace{3ex}
\formatRR{100100}{CLS}
\end{indented}\vspace{4ex}
\begin{indented}{\bf Description:}
The frame buffer and Z buffer are cleared.  Every half-word of the
color buffer is set to the low half-word of {\tt r1} and every
half-word of the Z buffer is set to the low half-word of {\tt r2}. 
This instruction is non-blocking and may return before the clear is
finished, but the {\tt TREEPUT} instruction will block if {\tt CLS}
has not yet terminated.
\end{indented}
\vspace{2em}

\noindent\textsf{\textbf{\Large TREESET}}\index{instruction set!TREESET@{\tt TREESET}}\par
\noindent {\sc Setup Tree and Clear Enable Flags}\par\begin{indented}{\bf Format:}
{\tt TREESET r1, r2, r3}\par\vspace{3ex}
\formatRRR{100101}{TREESET}
\end{indented}\vspace{4ex}
\begin{indented}{\bf Description:}
The tree is setup so that its left-most pixel will be at X coordinate
{\tt r1} and Y coordinate {\tt r2}.  The enable registers of the tree
are set to the complement of the lower 16 bits of {\tt r3}. 
Specifically, the {\tt O} registers will be set to their corresponding
bit in {\tt r3} and the {\tt I} registers will be set to the
complement of that.  Using {\tt r0} enables all of the pixels.  The
least significant bit corresponds to the right-most pixel in the tree.
The X coordinate {\tt r1} must be a multiple of 16.
\end{indented}
\vspace{2em}

\newpage
\noindent\textsf{\textbf{\Large TREEIN}}\index{instruction set!TREEIN@{\tt TREEIN}}\par
\noindent {\sc Edge Equation Test}\par\begin{indented}{\bf Format:}
{\tt TREEIN r1, r2}\par\vspace{3ex}
\formatRR{100110}{TREEIN}
\end{indented}\vspace{4ex}
\begin{indented}{\bf Description:}
The value in register {\tt r1} is placed at the root of the tree with
adder {\tt r2}.  After the data has trickled to the bottom of the
tree, the left-most leaf will contain {\tt r1}, the next leaf will
contain {\tt r1 + r2}, and so on until the right-mode leaf will
contain {\tt r1 + 15*r2}.  At the leaf, the {\tt I} register will stay
a 1 if the result is non-positive.  The {\tt O} register will stay a 0
if the result is non-negative.  Register {\tt r1} will then be
incremented by 16 times {\tt r2} to get it ready for the next call to
{\tt TREEIN}.
\end{indented}
\vspace{2em}

\noindent\textsf{\textbf{\Large TREEVAL}}\index{instruction set!TREEVAL@{\tt TREEVAL}}\par
\noindent {\sc Parameter Interpolation}\par\begin{indented}{\bf Format:}
{\tt TREEVAL r1, r2, imm}\par\vspace{3ex}
\formatRRI{100111}{TREEVAL}
\end{indented}\vspace{4ex}
\begin{indented}{\bf Description:}
The value in register {\tt r1} is placed at the root of the tree with
adder {\tt r2}.  After the data has trickled to the bottom of the tree,
the left-most leaf will contain {\tt r1}, the next leaf will contain
{\tt r1 + r2}, and so on until the right-mode leaf will contain
{\tt r1 + 15*r2}.  The constant {\tt imm} specifies what will happen
to the resulting data:
\begin{itemize}
\item[\bf 0]The high half-word of the result will be treated like a Z value and
will be compared with the Z value for that leaf's pixel.  If the computed
Z value is greater than the Z-Buffer's value, then the {\tt I} register
is cleared and the {\tt O} register is set.  The Z value is also
stored for later write to the Z-Buffer.
\item[\bf 1]Bits 16 through 20 of the result are stored in bits 11 through 15
of the color register at the leaf.  This will be displayed as the red
color when the color buffer is displayed.
\item[\bf 2]Bits 16 through 21 of the result are stored in bits 5 through 10
of the color register at the leaf.  This will be displayed as the green
color when the color buffer is displayed.  This component has one extra
bit because the eye is more sensitive to green than it is to red and blue.
\item[\bf 3]Bits 16 through 20 of the result are stored in bits 0 through 4
of the color register at the leaf.  This will be displayed as the blue
color when the color buffer is displayed.
\end{itemize}
Register {\tt r1} will then be incremented by 16 times
{\tt r2} to get it ready for the next call to {\tt TREEVAL}.
\end{indented}
\vspace{2em}

\newpage
\noindent\textsf{\textbf{\Large TREEPUT}}\index{instruction set!TREEPUT@{\tt TREEPUT}}\par
\noindent {\sc Tree Write to Frame Buffer}\par\begin{indented}{\bf Format:}
{\tt TREEPUT }\par\vspace{3ex}
\formatOP{101000}{TREEPUT}
\end{indented}\vspace{4ex}
\begin{indented}{\bf Description:}
At each leaf, the value {\tt I | !O} is calculated and used as the
enable register, and enabled leaves write their color and Z values to
their pixels.
\end{indented}
\vspace{2em}

