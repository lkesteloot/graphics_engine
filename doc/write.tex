\documentclass{article}
\usepackage{moreverb}
\title{Graphics Engine}
\author{\it Lawrence Kesteloot\\\it Rob Wheeler}
\begin{document}
\maketitle

\newenvironment{indented}[1]{\vspace{2ex}\noindent{\bf #1}\begin{list}{}{\setlength{\topsep}{0em}}\item}{\end{list}}

\section{Introduction}
This document characterizes the applications of the Graphics Engine (GE).
The GE has an optimized instruction set for transforming and rasterizing 3-D
graphics primitives for low-end machines.  The GE typically is
included in home or arcade video game machines.  A general-purpose CPU
and the GE share memory for communication.  The CPU calculates the
game dynamics and sets up a graphics data structure (a display list)
that the GE traverses and rasterizes in a pipeline arrangement.

A typical application is a game that uses 3-D graphics along with
sound and 2-D sprites (the latter two being supplied by other
hardware).  The application uses the GE to transform, light, and shade
the generated geometry.  A typical game might require up to
3,000 triangles per scene to achieve rendering detail comparable to
current 2-D games.  Thus, at 30 frames a second, roughly 100,000
triangles per second must be transformed and half of these must be lit
and rasterized.  (About half of the triangles will be backface-culled.)

The GE instruction set is optimized for four tasks:
\begin{enumerate}
\item transforming coordinates;
\item clipping;
\item calculating plane equations; and
\item rasterizing polygons.
\end{enumerate}
Only minimal 3-D features are supported---clipping,
diffuse lighting, gouraud shading, and Z-buffering.  The GE has no
anti-aliasing, texturing, transparency, or level-of-detail support.

This is a very specialized CPU, not only because it is optimized
for graphics, but because it is designed for low-end, inexpensive
graphics.  It does not provide the flexibility of high-end graphics
workstations, and the game programmer is expected to abide by
restrictions, such as keeping coordinates within the range of
fixed-point values.  When necessary, we simplified the hardware at the
expense of a heavier burden on the programmer.  Most game programmers
are used to programming in assembler and having no programming support
whatsoever, so ours are not unusual nor unexpected demands.

\begin{figure}
\begin{verbatimtab}
      +-----------+               +------------+
      |           |               |            |
      |    CPU    |               |     GE     |
      |           |               |            |
      +-----------+               +------------+
            |                           |
            +-------------+-------------+
                          |
                   +-------------+     +------------------+
                   |             |     |                  |
                   |     RAM     |-----|  Video hardware  |
                   |             |     |                  |
                   +-------------+     +------------------+
\end{verbatimtab}
\caption{Typical machine configuration.}
\label{fig:configuration}
\end{figure}

A typical machine configuration is shown in
Figure~\ref{fig:configuration}.
The CPU and the GE will communicate using display lists that will be
stored in shared memory.  The CPU will setup the display list for the next frame
while the GE will traverse and rasterize the display list for
the current frame.  Semaphores in memory will keep the
producer-consumer pipeline synchronized.

\section{Memory Requirements}

\subsection{Video Memory}

Video memory will be double-buffered.  Since the
output device of this machine is likely to be an NTSC television or a monitor
of comparable quality (in video arcades), a frame resolution of at
most $640\times480$ is required.  Such a low resolution will increase the
effective rasterization speed since triangles will be smaller.  A
colormap system is not acceptable since we need to interpolate color
for lighting, so an RGB scheme with 5 bits for red and blue and 6 bits for 
green is used.

A Z-buffer will be used for hidden surface removal, with 16 bits per
pixel.  The total video memory requirement is then $3\mbox{
buffers}\times640\mbox{ X-resolution}\times480\mbox{
Y-resolution}\times2\mbox{ bytes/pixel} = 1,843,200$ bytes.

\subsection{Display List Memory}

Each vertex needs to store three coordinates (x, y, z) of 16 bits each (6
bytes), either three color components or three normal components of 8
bits each (3 bytes), with three bytes to spare if packed into three
4-byte words.  The extra three bytes can store both extra data about
the vertex and control information for the display list (the op-code).

We expect to rasterize about 3,000 triangles per frame, with many of
those in mesh format (one vertex per triangle).  This will take
roughly $6000 \times 12$ bytes plus other information, such as transformation
matrices (which are few).  The total display list memory requirement
is then at most 100,000 bytes per display list, times two since
display lists are double-buffered.

The GE will access all memory in big-endian order, so it would be most
efficient if the CPU's endianess were the same.

\subsection{Program Memory}

The program running on the GE will be on-chip. Programs running on the
CPU will probably be at most be 100,000 bytes, plus 200,000 bytes for
data.

\subsection{Total}

The total external memory requirements are then $1.8 + 0.2 + 0.3 = 2.3$
megabytes.  This probably will be rounded up to 4 megabytes,
requiring at least 22 bits in address fields.

\section{Graphics Engine}

The graphics engine is a Harvard architecture chip with 1024 32-bit words of
on-chip program memory, 1024 words of on-chip data memory, access to
external RAM (with DMA access), 32 word-length registers, one program
counter, one status word, and a RISC instruction set specialized for
graphics processing.  It has no interrupts, no memory management unit,
and no cache.

\subsection{Registers}

The GE has 32 general-purpose registers that are word-length (32 bits).
Reading register 0 always returns 0 and writing to it has no effect.

\subsection{Address Space}

Memory locations 0 through 1023 address words in the on-chip data memory.
External RAM is read through special DMA instructions that copy blocks
of memory into internal data memory.  Video memory is accessed through
special rasterization instructions.
There is no access to the program counter or the status word.

\section{Rasterization}

We use the plane equation method of triangle rasterization, which means
that we test every pixel in the bounding box of the triangle, and
those that are inside have their parameters (color in this case)
interpolated.  The linear function $Ax + By + C$ is at the heart of
this method, where $x$ and $y$ are pixel coordinates and $A$, $B$, and $C$
are {\tt s15.16} fixed-point coefficients.

\section{Special-Purpose Hardware for Rasterization}

The inner loop of the plane equation rasterizer must perform the following
operations:

\begin{enumerate}
\item Test three planes to see if they have the same sign
\item If same sign, write color and Z to frame buffer
\item Recalculate R, G, B, Z, and three plane equations
\end{enumerate}

If the operations are performed with general-purpose RISC instructions,
only about 30,000 triangles will be possible with a 100MHz clock.
Instead, we rasterize 16 horizontal adjoining pixels at a time
using a PixelPlanes-like linear expression evaluation tree.

\section{Instruction Set Requirements}

Preliminary simulations of the hardware and software indicate that the
most common instructions are those that manipulate the rasterization
tree.  These instructions, then, should be optimized to do the most
amount of work in the least about of time.  For example, they will
automatically increment one of the operand registers by 16 times the
other operand register, so as to get the registers ready for the next
iteration of the loop.  This will save 7 instructions to increment
these registers.

Computing the plane equations for each triangle requires about 70
additions, 90 multiplications, and one reciprocal per triangle. 
Because of our choice of fixed-point data formats (s15.16), a
fixed-point multiply will be useful in addition to the integer
multiply.  Since the reciprocal is used infrequently,
it is not necessary to optimize it as heavily
as the other instructions.  A fast multiply would have a greater
impact on the rasterization speed.

A common operation, both in transformations and in lighting, is the
dot product.  A sequential 4-element vector dot product requires four
multiplications and 3 additions.  If we guess that additions take one
clock cycle and multiplications take ten (based on the MIPS R4000 RISC chip
specifications), a dot product will take 43 clock cycles, not counting
the work required to load the matrix row into registers.  Instead,
this common operation can be implemented as an instruction that
multiplies two vectors directly in data memory.  On-chip data memory
is likely to be as fast as registers, and the four multiplications can
occur simultaneously.  The additions can happen in two steps, leaving
the result in a register, for a total time of 12 clock cycles, without
any additional need for memory loads.

A graph of instruction frequencies estimates is attached.

\end{document}

\section{Instruction Frequencies}
  Examine display list DMA
  Decode display list instruction LD SHIFT LD JMP
  Unpack vertex 3 LD 3 SHIFT 3 ST
  Unpack normal 3 LD 3 SHIFT 3 ST
  Transform vertex by projection/model-view 
    First row DOT
    Second row DOT
    Third row DOT
    Fourth row DOT
  Transform normal by model-view (75)
    First row DOT
    Second row DOT
    Third row DOT
  Back-face test
    Dot normal with viewer DOT CMP
  Lighting 
    Normalize normal by scaling 3 MUL
    Dot normal with light DOT
    Multiply colors, normalize 3 MUL
  Clip 40 LD 40 ADD 40 CMP 40 BRANCH 30 MUL .1 RECP 
  Rasterize 
    Calculate plane equations 100 ADD 100 MUL 30 LD 30 ST
    Loop over scanlines 10 ADD 10 CMP 10 BRANCH 60 LD 60 ADD
    Rasterize scanline
      20 ADD 20 CMP 20 JNE
      60 TREEIN, TREEVAL
      20 TREECLR, SET, PUT
  
